\documentclass[11pt]{article}

\usepackage[utf8]{inputenc}
\usepackage[T1,T2A]{fontenc}
\usepackage{hyperref}


\begin{document}
\section*{Оформление домашнего задания}
Домашнее задание должно быть оформлено в виде pdf-файла или ipython-тетрадки с развернутыми ответами на все вопросы и описанием проделанных шагов. 



\section*{Домашнее задание 3 (до \textbf{3/2/17)}. Классификация имен}

В этом домашнем задании мы рассмотрим задачу бинарной классификации. Пусть дано два списка имен: мужские и женские имена. Требуется написать классификат

\begin{enumerate}
\item \textbf{[2 балла]} Составьте самостоятельно как минимум две коллекции текстов разных стилей (например, коллекция текстов в публицистическом стиле и коллекция текстов в научном стиле). Коллекции текстов должны быть достаточно большие (порядка 5000 токенов). Посчитайте количество токенов и типов в каждой коллекции.

\item \textbf{[5 баллов]} Используя любой морфологический процессор, который вам нравится (pymorphy2, mystem), определите к какой части речи относятся слова из каждой коллекции текстов. При помощи nltk.FreqDist() составьте частотные словари: часть речи – количество слов,  к ней относящихся.


\item \textbf{[3 балла]} Посчитайте коэффициент корреляции Спирмена для полученных на предыдущем шаге частот частей речи. На основании полученного значения, сделайте вывод: подтверждается ли гипотеза, сформулированная в задании? Если вы рассматривали больше двух стилей, можно ли утверждать, что один стиль больше похож на второй, чем на третий?

\end{enumerate}

\end{document}
