\documentclass{article}
\usepackage[T2A,T1]{fontenc}
\usepackage[utf8]{inputenc}
\usepackage[russian,english]{babel}
\usepackage{hyperref}
\usepackage{amsmath}

\begin{document}



\section*{\foreignlanguage{russian}{Домашее задание 2 (к 25/02/2016). Модель Word2vec}}

\foreignlanguage{russian}{Составьте достаточно большую коллекцию текстов на любом языке или используйте тексты отсюда}:
\url{https://www.kaggle.com/c/word2vec-nlp-tutorial/data}.

\begin{enumerate}
    \item \foreignlanguage{russian}{\textbf{[1 балла]} Обучите модель word2vec. Оцените время обучения модели, используя модуль time.}
    
    \foreignlanguage{russian}{Есть два варианта обучения модели: по отзывам целиком и с учетам границ предложений. В принципе, погрешностью, которая возникает в первом случае можно пренебречь, но если вы хотите учитывать границы предложений, то можно использовать \texttt{sent\_tokenize} из \texttt{nltk.tokenize}.}
    
    \item \foreignlanguage{russian}{\textbf{[2 балла]} Приведите 5-10 примеров использования \texttt{.most\_similar} для определения близких слов. Корректно ли они найдены? Являются ли синонимами исходного слова?}
    \item \foreignlanguage{russian}{\textbf{[2 балла]} Приведите 5-10 примеров использования \texttt{.most\_similar}  для определения ассоциаций (А к Б, как В к?). Корректно ли найдены ассоциации?}
    \item \foreignlanguage{russian}{\textbf{[2 балла]} Приведите 5-10 примеров использования \texttt{.doesnt\_match} для определения лишнего слова. Корректно ли найдены лишние слова?}
    \item \foreignlanguage{russian}{\textbf{[3 балла]} Попробуйте найти такие пары и тройки слов, для которых }
    \begin{itemize}
    \item \foreignlanguage{russian}{\textbf{не} выполняются свойства коммутативности и транзитивности относительно операции определения близких слов.} 
    \item \foreignlanguage{russian}{выполняются свойства коммутативности и транзитивности относительно операции определения близких слов.} 
	
    \end{itemize}

\foreignlanguage{russian}{Обозначим отношение ``входить в топ-3 по \texttt{.most\_similar}''  символом $\circ$.}

\foreignlanguage{russian}{\textbf{Коммутативность}} $x \circ y \implies y \circ x$


\foreignlanguage{russian}{\textbf{Транзитивность} $x\circ y, y \circ z \implies x \circ z$  }

\end{enumerate}

\end{document}
