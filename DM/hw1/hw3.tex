\documentclass[11pt]{article}

\usepackage[utf8]{inputenc}
\usepackage[T1,T2A]{fontenc}
\usepackage{hyperref}


\begin{document}
\section*{Оформление домашнего задания}
Домашнее задание должно быть оформлено в виде pdf-файла или ipython-тетрадки с развернутыми ответами на все вопросы и описанием проделанных шагов. Данные к домашнему заданию можно найти на wiki.cs.hse.ru.



\section*{Домашнее задание 3 (до \textbf{09/03/17)}. Классификация имен}

В этом домашнем задании мы рассмотрим задачу бинарной классификации. Пусть дано два списка имен: мужские и женские имена. Требуется разработать классификатор, который по данному имени будет определять мужское оно или женское. 

\begin{enumerate}
\item \textbf{[1 балл]} Предварительная обработка данных: 1) удалите неоднозначные имена (те имена, которые являются и мужскими, и женскими одновременно), если такие есть; 2) создайте тестовое множество по следующему принципу: 20\% от общего количества имен на каждую букву (т.е. 20\% от имен на букву  А, 20\% имен на букву B, и.т.д.).


\item \textbf{[4 балла]} Используйте метод наивного Байеса для классификации имен: в качестве признаков используйте символьные $n$-граммы. Сравните результаты, получаемые при разных $n=2,3,4$ по $F$-мере и аккуратности. В каких случаях метод ошибается?

Для генерации $n$-грамм используйте \texttt{from nltk.util import ngrams}.


\item \textbf{[4 балла]} Используйте сеть с двумя слоями LSTM для определения пола. Представление имени для классификации в этом случае: 2-мерный бинарный вектор количество букв в алфавите $\times$ максимальная длина имени. Обозначим его через $x$. Если первая буква имени a, то $x[1][1] = 1$, если вторая – b, то  $x[2][1] = 1$. Не забудьте про регуляризацию нейронной сети дропаутами. Если совсем не получается запрограммировать нейронную сеть самостоятельно, обратитесь к туториалу тут: \url{https://github.com/divamgupta/lstm-gender-predictor/blob/master/train\_genders.py}. Сравните результаты, получаемые при разных значениях дропаута, разных числах узлов на слоях нейронной сети по $F$-мере и аккуратности. В каких случаях нейронная сеть ошибается?


\item \textbf{[1 балл]} Сравните результаты классификации разными методами. Какой метод лучше и почему?

\end{enumerate}

\end{document}
